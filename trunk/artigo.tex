\documentclass[12pt]{article}

\usepackage{sbc-template}

\usepackage{graphicx,url}

%\usepackage[brazil]{babel}   
\usepackage[latin1]{inputenc}  

\renewcommand{\refname}{Refer�ncias}

     
\sloppy

\title{\textit{AccTrace}: Rastreabilidade dos Requisitos de Acessibilidade desde as Fases de Engenharia de Requisitos At� a Codifica��o do \textit{Software}}

\author{Rodrigo Gon�alves de Branco\inst{1}, Maria Istela Cagnin\inst{1}, Debora Maria Barroso Paiva\inst{1}}


\address{
  Faculdade de Computa��o\\
  Universidade Federal de Mato Grosso do Sul (UFMS) \\
  Campo Grande, MS -- Brasil
  \email{\{rodrigo.g.branco,istela,dmbpaiva\}@gmail.com}
}

\begin{document} 

\maketitle

\begin{abstract}
  This meta-paper describes the style to be used in articles and short papers
  for SBC conferences. For papers in English, you should add just an abstract
  while for the papers in Portuguese, we also ask for an abstract in
  Portuguese (``resumo''). In both cases, abstracts should not have more than
  10 lines and must be in the first page of the paper.
\end{abstract}
     
\begin{resumo}
\end{resumo}

\section{Introdu��o}

Fornecer \textit{softwares} acess�veis continua sendo um desafio nos dias atuais, com diversas pesquisas na �rea \cite{lazar:04,brajnik:06,zeng:05}. Dentre as dificuldades pertinentes do problema, podemos destacar a identifica��o dos requisitos de acessibilidade e sua posterior propaga��o e rastreabilidade at� a fase de constru��o do produto. Enquanto existem propostas para integrar usabilidade e acessibilidade nos processos de Engenharia de \textit{Software}, muitos desenvolvedores n�o sabem como implementar tais produtos acess�veis \cite{1630123,alves:11}.

A utiliza��o de ferramentas CASE nos processos de Engenharia de \textit{Software} � muito comum, em geral, aumentam a produtividade dos desenvolvedores, j� que estas ferramentas automatizam algumas tarefas diminuindo o esfor�o e o tempo de constru��o da solu��o. Na �rea de acessibilidade, � poss�vel encontrar diversas ferramentas especializadas, como \textit{frameworks}, simuladores, validadores, entre outras. Contudo, os desenvolvedores n�o est�o satisfeitos com o suporte fornecido por estas ferramentas \cite{Trewin:2010:ACT:1805986.1806029}.

Muitas das ferramentas descritas normalmente s�o utilizadas quando o produto final j� est� pronto. Por isso, seria interessante que os requisitos de rastreabilidade, assim que localizados, pudessem ser rastreados para identificar se est�o sendo codificados corretamente. H� v�rios estudos relacionados a rastreabilidade de requisitos de forma gen�rica \cite{5970169,292398,5485417,6405269}, por�m poucos relacionados aos requisitos de acessibilidade durante o processo de desenvolvimento de \textit{software} \cite{analuizadias:2010}.

Este trabalho apresenta o \textit{AccTrace}

\section{Caracter�sticas da Ferramenta}

\section{Arquitetura da Ferramenta}

\section{Prova de Conceito}

\section{Conclus�es}



\bibliographystyle{sbc}
\bibliography{sbc-template}

\end{document}
